% THIS IS SIGPROC-SP.TEX - VERSION 3.1
% WORKS WITH V3.2SP OF ACM_PROC_ARTICLE-SP.CLS
% APRIL 2009
%
% It is an example file showing how to use the 'acm_proc_article-sp.cls' V3.2SP
% LaTeX2e document class file for Conference Proceedings submissions.
% ----------------------------------------------------------------------------------------------------------------
% This .tex file (and associated .cls V3.2SP) *DOES NOT* produce:
%       1) The Permission Statement
%       2) The Conference (location) Info information
%       3) The Copyright Line with ACM data
%       4) Page numbering
% ---------------------------------------------------------------------------------------------------------------
% It is an example which *does* use the .bib file (from which the .bbl file
% is produced).
% REMEMBER HOWEVER: After having produced the .bbl file,
% and prior to final submission,
% you need to 'insert'  your .bbl file into your source .tex file so as to provide
% ONE 'self-contained' source file.
%
% Questions regarding SIGS should be sent to
% Adrienne Griscti ---> griscti@acm.org
%
% Questions/suggestions regarding the guidelines, .tex and .cls files, etc. to
% Gerald Murray ---> murray@hq.acm.org
%
% For tracking purposes - this is V3.1SP - APRIL 2009

\documentclass{edm_template}

\begin{document}

\title{Skill-Specific Impacts: Examining Item-Level Treatment Effects
  from the Cognitive Tutor Effectiveness Study}
%\subtitle{[Extended Abstract]
%\titlenote{A full version of this paper is available as
%\textit{Author's Guide to Preparing ACM SIG Proceedings Using
%\LaTeX$2_\epsilon$\ and BibTeX} at
%\texttt{www.acm.org/eaddress.htm}}}
%
% You need the command \numberofauthors to handle the 'placement
% and alignment' of the authors beneath the title.
%
% For aesthetic reasons, we recommend 'three authors at a time'
% i.e. three 'name/affiliation blocks' be placed beneath the title.
%
% NOTE: You are NOT restricted in how many 'rows' of
% "name/affiliations" may appear. We just ask that you restrict
% the number of 'columns' to three.
%
% Because of the available 'opening page real-estate'
% we ask you to refrain from putting more than six authors
% (two rows with three columns) beneath the article title.
% More than six makes the first-page appear very cluttered indeed.
%
% Use the \alignauthor commands to handle the names
% and affiliations for an 'aesthetic maximum' of six authors.
% Add names, affiliations, addresses for
% the seventh etc. author(s) as the argument for the
% \additionalauthors command.
% These 'additional authors' will be output/set for you
% without further effort on your part as the last section in
% the body of your article BEFORE References or any Appendices.

\numberofauthors{2} %  in this sample file, there are a *total*
% of EIGHT authors. SIX appear on the 'first-page' (for formatting
% reasons) and the remaining two appear in the \additionalauthors section.
%
\author{
% You can go ahead and credit any number of authors here,
% e.g. one 'row of three' or two rows (consisting of one row of three
% and a second row of one, two or three).
%
% The command \alignauthor (no curly braces needed) should
% precede each author name, affiliation/snail-mail address and
% e-mail address. Additionally, tag each line of
% affiliation/address with \affaddr, and tag the
% e-mail address with \email.
%
% 1st. author
\alignauthor
Adam C Sales\\
       \affaddr{University of Texas College of Education}\\
       \affaddr{Austin, TX, USA}\\
       \email{asales@utexas.edu}
% 2nd. author
\alignauthor
John F Pane\\
       \affaddr{RAND Corporation}\\
       \affaddr{Pittsburgh, PA, USA}\\
       \email{jpane@rand.org}
}



\maketitle
\begin{abstract}
This paper provides a sample of a \LaTeX\ document which conforms to
the formatting guidelines for ACM SIG Proceedings.
It complements the document \textit{Author's Guide to Preparing
ACM SIG Proceedings Using \LaTeX$2_\epsilon$\ and Bib\TeX}. This
source file has been written with the intention of being
compiled under \LaTeX$2_\epsilon$\ and BibTeX.

The developers have tried to include every imaginable sort
of ``bells and whistles", such as a subtitle, footnotes on
title, subtitle and authors, as well as in the text, and
every optional component (e.g. Acknowledgments, Additional
Authors, Appendices), not to mention examples of
equations, theorems, tables and figures.

To make best use of this sample document, run it through \LaTeX\
and BibTeX, and compare this source code with the printed
output produced by the dvi file.
\end{abstract}


\keywords{ACM proceedings, \LaTeX, text tagging} % NOT required for Proceedings

\section{Introduction}

\section{The CTA1 Effectiveness Study}
\section{Why Look at Individual Items?}

\section{Methodology: Two Approaches to Estimation}
\subsection{Design-Based Estimation}
At its essence, estimating a treatment effect for an individual item
is the same as estimating any other treatment effect.
The 32 items of the Algebra I posttest are simply 32 different
outcomes, calling for 32 different, standard analyses.

This approach allows for design-based estimation
(e.g. \cite{dunning2010design}, \cite{freedman2008regression})---that is, estimation
that does not rely on any modeling assumptions, but only the design
itself.
Design-based estimators take treatment assignment to be the only
source of randomness, modeling all other variables as fixed.
Since the probability distribution of the treatment assignment is a function of
the experimental design, it is a known quantity and need not be
assumed.
The CTA1 effectiveness trial used a blocked, clustered design, with
schools (not students) randomized to treatment or control conditions,
within pre-set pairs.
In this analysis, we used the estimator for blocked clustered designs
proposed in \cite{schochet2015statistical} and implemented in the
RCT-YES software.
That is, we estimate the effect of assignment to the CTA1 condition on
item $i$ as
\begin{equation}
\hat{\tau}_i^{db}=\frac{1}{n_p}\displaystyle\sum_{p=1}^{n_p}
\bar{Y}_i^{p1}-\bar{Y}_i^{p0}
\end{equation}
where $p=1,\dots,n_p$ indexes matched pairs, and $\bar{Y}_i^{p1}$ and
$\bar{Y}_i^{p0}$ are proportions of students who answered item $i$
correctly in the school assigned to treatment and the school assigned
to control, respectively.
$\tau_i^{db}$ is guarenteed to be unbiased, and its associated
variance estimate is conservative in expectation.
RCT-YES also provides an option for adjustment of the estimator for a
baseline covariate (we used pretest scores) which is assymptotically
unbiased.

\subsection{Model-Based Estimation and Partial Pooling}

The design-based estimates $\hat{\tau}^{db}$ avoid statistical
modeling, but in doing so often sacrifice statistical precision.
In particular, the items of the Algebra post-test are inter-related,
and the 32 separate design-based estimates ignore that structure.
An alternative modeling strategy begins with the premise that data
from one post-test item may be informative regarding other items.
An estimator that partially pools (c.f. \cite{gelmanHill}) data across
items---exploiting data from the entire post-test in each of the 32
analyses---can achieve greater precision that one that looks at each
item separately.
One such estimator is mixed-effects (or multi-level) logistic
regression.
Our parametrization began with a multi-level explantory Rasch modlel
(e.g. \cite{explantoryIRT}):
\begin{equation}\label{eq:rasch2}
Pr(Y_{ij}=1)=logit^{-1}(\eta^i_j-\delta_i)
\end{equation}
where $logit^{-1}$ is the inverse-logit function,
$logit^{-1}(x)=exp(x)/[1+exp(x)]$, $\eta^i_j$ is an ``ability''
parameter, encoding student $j$'s ability to answer item $i$
correctly, and $\delta_i$ is a problem difficulty parameter, encoding
the difficulty of correctly answering problem $i$.

Model

\section{Results: What did CTA1 Help With?}
\section{Discussion}

\subsection{Acknowledgments}
\subsection{References}

\end{document}
